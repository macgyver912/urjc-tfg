\chapter{Introducción}
\label{sec:intro}


\section{Contexto}
\label{sec:contexto}

En la actualidad, uno de los principales desafíos a los que se enfrenta la gestión del espacio aéreo es el intenso y creciente tráfico de aeronaves, especialmente saturado en ciertas áreas estratégicas y principales aeropuertos de cada país. En estas áreas, el Control de Tráfico Aéreo (ATC, \textit{Air Traffic Control}) es un auténtico desafío para garantizar la seguridad y calidad de las operaciones aéreas.

Además, las estrategias de optimización de las operaciones aéreas junto con los planes de reducción de emisiones de efecto invernadero promueven la búsqueda de rutas cada vez más eficientes, lo que conlleva en ciertos casos a acortar trayectorias y a concentrar más aún el tráfico en determinadas áreas.

Es por ello que que, cada vez más, se emplean herramientas informáticas que ayudan a evaluar y mejorar la calidad de las operaciones aéreas mediante simulaciones computacionales.

//TODO: seguir describiendo el contexto en el que se enmarca el TFG

\newpage

\section{Motivación}
\label{sec:motivación}

En este contexto, se plantea el desarrollo de una herramienta de simulación de control de tráfico aéreo que permita, por un lado, acercar a curiosos del mundo de la aviación al conocimiento de las bases del Control de Tráfico Aéreo y, por otro lado, que sirva como generador de datos para un posterior análisis y tratamiento de los mismos enfocado en la evaluación de las rutas existentes y el estudio de nuevas rutas durante su diseño.

Para ello se plantea un sistema que sea capaz de englobar el propio entorno de simulación interactivo con las herramientas de diseño de nuevos aeropuertos y rutas de una forma ágil y sin requerir unos profundos conocimientos de informática ni de software de gestión.

De esta forma se podrán evaluar puntos clave en el diseño de nuevas rutas de aproximación a aeropuertos como la trayectoria de las aeronaves, la necesidad de descensos escalonados o descensos continuos, cumplimiento de servidumbres aeroportuarias y la viabilidad de operaciones paralelas en aeropuertos que así dispongan sus pistas.

//TODO: seguir describiendo la motivación



\section{Estructura de la memoria}
\label{sec:estructura}

\textit{[1 página] Definir en items cada capítulo del proyecto, incluyendo el capítulo de conclusiones}